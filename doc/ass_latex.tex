\documentclass[10pt,A4,makeidx]{article}
\usepackage{color}
\usepackage{amsmath}
\usepackage{float}
\usepackage{subfig}
\usepackage{comment}
\UseRawInputEncoding

\textwidth 16.5cm
\hoffset -2.0cm
\textheight 23cm
\voffset -1.5cm

\newcommand{\tR}{\texttt{R}}
\newcommand{\adist}{\overset{\cdot}{\underset{\cdot}{\sim}}}
%\setcounter{tocdepth}{2}

\title
{Effects of Structural Characteristics on House Prices}
\author{Sean Peralta Garcia (23088091)}
\date {}

\usepackage{Sweave}
\begin{document}

\maketitle


\section{Executive Summary}
\section{Introduction}
  Owning a house is one of the biggest investments a person can make. As such, it 
  is of great interest to prospective buyers, sellers and lenders to accurately
  predict the price of a home. There are a range of models and techniques that 
  attempt to predict the price of a home, from observational appraisals all 
  the way to machine learning models\cite{hedonic_ai}. One method widely studied is that of
  hedonic pricing. A Hedonic Pricing Model (henceforth, HPM) is a model that 
  attempts to estimate the price of a good by taking its observable characteristics
  and then weighting them according to their relative impact on the price. Models
  utilise a range of measures that can be categorised into several groups, 
  namely structural, neighbourhood, and environmental. \cite{hedonic_rev} Hearth and Maier,
  in a literature review of HPMs for real estate, had identified that the 
  neighbourhood and environmental factors were generally over-researched. While
  social factors and the "implicit value of structural characteristics" was 
  under-researched.\cite{lit_rev}

  \subsection{Related Work}
  There appears to be a consensus that creating a HPM, particularly focussing on
  structural characteristics, has problems with heteroskedasticity. This means
  that linear models may not be entirely appropriate to estimate the response.
  This was sought to be relieved by Selim, Limsombunchai and Malpezzi by using
  a semi-logarithmic form wherein the response variable is transformed by the natural log.
  \cite{hedonic_ai, hedonic_rev, hedonic_regress} Additionally, Malpezzi explains
  that this effectively allows value added to the house to be proportional to 
  other variables in the model and for an easier interpretation of coefficients
  such that the coefficient of a measure is the percentage change for 1 unit 
  difference in the measure.\cite{hedonic_rev}
    
  \subsection{Data Set}
  
  \subsection{Aim}
  Build 
\section{Methodology}
\section{Results}
\section{Discussion}

\bibliographystyle{apalike}
\begin{thebibliography}{9}
\bibitem{lit_rev}
Herath, S. K. Maier, G. (2010). The hedonic price method in real estate and housing market research. A review of the literature.. Institute for Regional Development and Environment (pp. 1-21). Vienna, Austria: University of Economics and Business.

\bibitem{hedonic_ai}
Limsombunchai, V. (2004). House price prediction: Hedonic price model vs. artificial neural network. New Zealand Agricultural and Resource Economics Society Conference, 25-26 June 2004. Blenheim, New Zealand: New Zealand Agricultural and Resource Economics Society.

\bibitem{hedonic_rev}
Malpezzi, S. (2003). Hedonic pricing models: a selective and applied review. Housing economics and public policy, 1, 67-89.

\bibitem{hedonic_regress}
Selim, S. (2008). DETERMINANTS OF HOUSE PRICES IN TURKEY: A HEDONIC REGRESSION MODEL . Dogus Universitesi Dergisi , 9 (1) , 65-76 . Retrieved from 
\end{thebibliography}

\end{document}
